A\-W\-E\-M\-C is programmed using C++ and relies on \href{http://www.qt-project.org/>}{\tt Qt}, \href{http://curl.haxx.se/>}{\tt c\-U\-R\-L}, and (eventually) \href{http://www.sfml-dev.org/>}{\tt S\-F\-M\-L}. \href{http://jsoncpp.sourceforge.net/>}{\tt Json\-Cpp} is included in the source code.

There are 7 central concepts you should understand\-:


\begin{DoxyItemize}
\item Media items are organized in a folder structure using J\-S\-O\-N files to hold metadata and such.
\item Media players play media files.
\item Media services are standalone applications, like a web browser, Netflix, or i\-Tunes.
\item Metadata scrapers get relevant information and images for a media item.
\item Media types describe the default metadata values for a collection of items, like Movies or T\-V Shows.
\item Folder generators make a folder structure with organization based on things like Genre for a specific media type. They also use metadata scrapers to collect information about media items to be placed in the folder structure.
\item Skins describe the way the \hyperlink{namespace_u_i}{U\-I} looks and give default images to files.
\end{DoxyItemize}

\subsection*{Namespaces}

There are a few namespaces you should know about\-:


\begin{DoxyItemize}
\item {\ttfamily std} -\/ Well, this one should be obvious.
\item {\ttfamily \hyperlink{namespace_json}{Json}} -\/ This is the namespace for \href{http://jsoncpp.sourceforge.net/>}{\tt Json\-Cpp}.
\item {\ttfamily \hyperlink{namespace_a_w_e}{A\-W\-E}} -\/ All backend classes fall under this namespace.
\item {\ttfamily \hyperlink{namespace_u_i}{U\-I}} -\/ All user interface classes (i.\-e. windows and widgets) fall under this.
\item {\ttfamily T\-C\-O\-M} -\/ My own little colored command line library included with the code for debugging (it will be removed later). 
\end{DoxyItemize}