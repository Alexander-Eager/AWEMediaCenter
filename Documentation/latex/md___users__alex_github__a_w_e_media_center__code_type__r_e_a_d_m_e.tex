The information for a media type is kept in a J\-S\-O\-N file\-: \begin{DoxyVerb}{
    "type": <"folder" or "file">,

    "metadata": {
        "type": <type name>,
        "name": "",
        "location": "",

        "description": "",

        "icons": {
            "default": 0,
            "files": []
        },

        "fanarts": {
            "default": 0,
            "files": []
        },

        "details": {
            "_order": [<first detail>, <second detail>, ...],

            <first detail>: <default value>,
            <second detail>: <default value>,
            // more default details attributes
        }
    },

    "matching files": [
        <file expression you could use in terminal>,
        // more file expressions that also match
    ]

    "contains": [
        // nested media types
    ]
}
\end{DoxyVerb}


\section*{Metadata Default Values}

Sometimes, a property cannot be given a value. In these instances, the value should be specified as the default value. Specified by {\ttfamily \char`\"{}\char`\"{}} or {\ttfamily \mbox{[}\mbox{]}}.

A metadata scraper can decide to inherit a property from the parent folder, which is another kind of default property, but that is not explicity defined in type or file J\-S\-O\-N configurations.

\section*{Folder Structure Generation}

\subsection*{Explanation of File Organization}

The {\ttfamily \char`\"{}contains\char`\"{}} tag indicates how the files should be organized by the folder generator. To better explain this, take T\-V episodes. The episodes should be organized by series and then season. So the {\ttfamily \char`\"{}contains\char`\"{}} property of the {\ttfamily \char`\"{}\-T\-V Series\char`\"{}} type would contain the {\ttfamily \char`\"{}\-T\-V Season\char`\"{}} type, which would contain the {\ttfamily \char`\"{}\-T\-V Episode\char`\"{}} type.

Then, when the folder is generated, the folder structure would look like this\-: \begin{DoxyVerb}All Items
├── <name of series>
│   ├── <season #>
│   │   └── <Episodes>
│   └── <season #>
│       └── <Episodes>
└── <name of series>
    ├── <season #>
    │   └── <Episodes>
    └── <season #>
        └── <Episodes>
\end{DoxyVerb}


You can instantiate the sub-\/type only if you so desire. You can instantiate the {\ttfamily \char`\"{}\-T\-V Episode\char`\"{}} type by itself. Because of this, you should list the properties from the hierarchy, like {\ttfamily \char`\"{}\-Season\char`\"{}} and {\ttfamily \char`\"{}\-Series\char`\"{}}, in the {\ttfamily \char`\"{}metadata.\-details\char`\"{}} tag.

\subsection*{Root directories for the type}

In addition to the {\ttfamily All Items} folder described above, there are also folders for every member of {\ttfamily \char`\"{}metadata.\-details\char`\"{}}. If, for example, there were a {\ttfamily \char`\"{}metadata.\-details.\-Genre\char`\"{}}, the following folder structure would be generated\-: \begin{DoxyVerb}Genre
├── Genre Value 1
│   └── Files with that value
└── Genre Value 2
    └── Files with that value
\end{DoxyVerb}


If the detail is a string, integer, or boolean, it is displayed directly as one of the values. If the detail is an array, each element in that array is a value. So if a movie has two directors, {\ttfamily Bob} and {\ttfamily Bill}, that movie appears under both {\ttfamily Bob} and {\ttfamily Bill}. {\ttfamily Bob} and {\ttfamily Bill} are separate folders under the {\ttfamily Director} folder. Also note that the actual property name is {\ttfamily Director(s)}, but the folder name is {\ttfamily Director}. If a section of text is contained in {\ttfamily ()}, {\ttfamily \{\}}, or {\ttfamily \mbox{[}\mbox{]}} in a detail's name, that text along with the {\ttfamily ()}, {\ttfamily \{\}}, or {\ttfamily \mbox{[}\mbox{]}}, is removed. In addition, other special characters that are not allowed in file names are removed individually. When defining a new type, you should name your detail properties carefully to avoid awkward folder names.

All generated folder structures have an {\ttfamily All Items} folder that lists all of the items. 